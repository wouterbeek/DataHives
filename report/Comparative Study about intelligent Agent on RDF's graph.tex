\documentclass{article}
\usepackage[utf8]{inputenc}
\usepackage[T1]{fontenc}
\usepackage[english]{babel}
\usepackage{enumitem}
\usepackage{color}
\usepackage[colorlinks,linkcolor=grey, urlcolor=gray, citecolor=red, breaklinks, pagebackref]{hyperref}

\definecolor{gray}{rgb}{0.7,0.7,0.7}
\definecolor{grey}{rgb}{0.3,0.5,0.5}
\newenvironment{itemh}[0]{\begin{itemize}[label=$\heartsuit$, font=\color{gray} \small]}{\end{itemize}}
\newenvironment{itemH}[0]{\begin{itemize}[label=$\heartsuit$, font=\color{gray} \large]}{\end{itemize}}

\begin{document}
\title{Comparative Study about intelligent agent on RDF's graph\\
	\textit{Bachelor Project in Semantic Web using bee and ant agent}}
\author{	Baudouin Duthoit\\
		Student number : 2540566
	\and
		Wouter Beek
	\and
		Stefan Schlobach
	}
\maketitle

\abstract{
	RDF is a standard of W3C to enable computer and human to understand data, to give them sense.
	That is the Semantic Web.
	Data is stored in triples formed of a subject and a predicate and an object.
	Those triples themselves are on different computers well spread all over the web.
	More and more people are using the Semantic Web and so the online data is increasing a lot
	making the reasoning over the graph really complex and the more complex it is,
	the more we need efficient agents to increase performances.
	Our goal is to create agents that performs that reasoning base on the ants behaviour
	and compare them to the bee agents we already have, created last year by Christophe Guéret.
	Those agents are coded in JavaScript and perform as a Web application that any client can use,
	an example of the preexisting project can be found here \url{http://wouterblog.com/} .
}

\newpage

\tableofcontents

\newpage

\section{Hypothesis}
	The goal of this project is to create a new kind of agents based on a different behaviour,
	agent that we think more powerful than the previous one.

\section{Evaluation}
	We will verify the hypothesis via different criteria.
	The first way of compare two agents is quantitative using the number of discovered node (entailment regime).
	This one is relatively simpler to evaluate than the qualitative which require a human subjective interaction (user feedback).
	Those two methods will be used to compare 3 kind of agents:
	\begin{itemh}
		\item Random (base)
		\item Bee (already implemented)
		\item Ant (to be implemented)
	\end{itemh}

\section{Planning}
	\subsection{Estimated tasks}
		Some high-level tasks, to be filled in later:
		\begin{itemH}
			\item Implementation framework:
				\begin{itemh}
					\item Have a look at the JS-based triple store rdfstore-js
					\item Integrate Pepijn’s work (as much as possible)  into the new implementation.
					\item Visualization (optional component) uses D3JS.
				\end{itemh}
			\item Strategies:
				\begin{itemh}
					\item Random movement + no communication.
					\item Bee: random node choice / communication / local traversal.
					\item Ant: movement based on pheromone values + communication via environment/graph annotation.
				\end{itemh}
			\item Evaluation:
				\begin{itemh}
					\item Quantify over the number of deductions (over time) achieved by a group of agents.
					\item Qualitative evaluation: more difficult…
				\end{itemh}
		\end{itemH}

\section{Work in Progress}
	\subsection{First try with RDFstore-js}
		\paragraph{} The first attempt to implement those agents was to use JS, based on the project :
		\url{https://githu.com/antoniogarrote/rdfstore-js}.
		Unfortunately, it has been misleading because the documentation was lacking
		and implementing functions was thus hard to do.
		It took me 3 days to understand how to load a store in local to work on it !
	\subsection{Work on DataHives \#1}
		\paragraph{} This is a submodule of the project Pragmatic Semantic.
		It already has random agents implemented.
		Those agents are defined by a cycle of action they perform : Navigate / Act / Communicate.
		The random agent go somewhere random, don't act nor communicate.
		We will implement naviguation, action and communication functions step by step,
		adding a few feature at a time to be sure to get something working in the end.
		The web interface will show where did the agents went (could be useful for qualitative evalutation).
		\paragraph{} To start the local server, we had to launch a command from the Pragmatic Semantic folder.
		The purpose of this command was to laucnh the local server.
		Then, on it, we could start some random agents using the simple dh\_test(\_) in prolog.
		After that, the visualization of the agents and their progress is at the web adresse localhost:3040.
		Unfortunately, the heavy structure was quite hard to manipulate correctly and didn't make a stand alone system.
		That why DataHives has been recast.
	\subsection{DataHives \#2 recast}
		\paragraph{} The recast makes the system more easy to use and more stand alone.

\newpage

\section{Literature}
	Papers :
	\begin{itemh}
		\item Hayes2014 RDFS 1.1 Semantics (section 9.2.1 gives a fair overview).
		\item Pepijn Kroes, 2013, BNAIC poster.
		\item Kathrin Dentlez, Christophe Gueret, Stefan Schlobach, IEEE full paper.
	\end{itemh}
	Other links :
	\begin{itemh}
		\item\url{http://www.w3.org/TR/2014/REC-rdf11-mt-20140225/#rdfs-entailment}
		\item \url{http://www.doc.gold.ac.uk/~mas02gw/prolog_tutorial/prologpages/index.html#menu}
	\end{itemh}
\end{document}
